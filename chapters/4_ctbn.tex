\chapter{Continuos Time Bayesian Networks}
  \label{chapter_ctbn}
  \section{Introduzione}
  \paragraph{}
  In questo capitolo viene presentato il modello delle \textbf{Continuous Time Bayesian Networks (CTBN)} e la sua semantica.\\
  Questo modello è costruito sulla base dei concetti esposti in relazione ai \textit{Processi Markoviani a Tempo Continuo},
  riprendendo inoltre la definizione delle \textit{Bayesian Networks} per sfruttarne la rappresentazione compatta.
  
  \paragraph{}
  Nonostante anche le \textit{Dynamic Bayesian Networks} consentano di modellare una rete che progredisce
  nel tempo, tracciandone lo stato ad istanti successivi, esse non rappresentano esplicitamente il concetto del tempo. 
  Nelle \textit{DBN}, infatti, le variabili sono indicizzate stabilendo un ordinamento, ma non è presente alcun riferimento 
  all'istante temporale effettivo a cui esse appartengono. 
  Conseguentemente, tale prerogativa fa sì che nelle reti dinamiche il tempo venga regolarmente suddiviso in intervalli di ampiezza costante, 
  comportando limitazioni significative all'efficacia e all'applicabilità di tale modello, precludendo ad esempio la possibilità che i diversi processi 
  evolvano con granularità temporali differenti, oppure che le osservazioni avvengano ad intervalli irregolari.\\
  L'elusione di queste limitazioni richiederebbe di rappresentare l'intero sistema con la più piccola granularità
  possibile, implicando quindi di includere anche istanti temporali intermedi in cui non è occorsa nessuna
  osservazione.\\
  Le CTBN consentono perciò di superare tali limiti, basandosi sui processi Markoviani ma mantenendo
  le idee fondanti delle \textit{DBN}.

  \begin{definition}[Continuous Time Bayesian Networks] \cite{nodelman-2002}
    Sia \textit{\textbf{X}} un insieme di variabili $X_1, X_2, ..., X_n$, dove ciascuna variabile rappresenta
    un processo Markoviano a tempo continuo.\\
    Una Continuous Time Bayesian Network $\mathcal{N}$ su \textit{\textbf{X}} è definita da:
    \begin{itemize}
      \item Una distribuzione iniziale $\mathcal{B}_0$, rappresentata come rete Bayesiana
      \item Un modello di transizione, costituito da:
      \begin{itemize}
        \item Un grafo $\mathcal{G}$ sulle variabili in \textit{\textbf{X}}
        \item Una \textit{Conditional Intensity Matrix} $Q_{X|Pa(X)}$ per ciascuna variabile in \textit{\textbf{X}}
      \end{itemize}
    \end{itemize}
  \end{definition}

  \paragraph{}
  Come già anticipato in precedenza nel contesto dei \textit{Conditional Markov Processes},
  la presenza di un arco da X ad Y rappresenta la dipendenza delle intensità di transizione
  di Y dai valori assunti da X.\\
  È importante sottolineare come nelle CTBN, a differenza delle reti Bayesiane classiche,
  non sia necessario imporre che il grafo sia aciclico.
  Infatti, anche se esiste un arco \textit{(X, Y)} che denota l'influenza di Y nell'evoluzione di X,
  è concettualmente ammissibile che la transizione di Y possa dipendere a sua volta da X,
  implicando l'esistenza di un arco \textit{(Y, X)}.  

  \section{Campionamento}
  \paragraph{}
  Il modello delle CTBN prevede un'assunzione fondamentale, la quale impone che una ed una sola variabile 
  possa effettuare una transizione verso un nuovo stato in uno specifico istante temporale. 
  Questo vincolo ha un'importante implicazione dal punto di vista semantico, dal momento che può essere visto come 
  la formalizzazione del fatto che ciascuna variabile riflette un aspetto della realtà atomico e distinto rispetto alle altre variabili.
  Non sarebbe quindi né significativo, né realistico, usare le CTBN per modellare un dominio in cui più variabili possono cambiare stato nello stesso istante.
  
  \paragraph{}
  Dopo aver introdotto tutte le definizioni e i concetti teorici necessari, viene ora mostrata,
  in forma di pseudo-codice, la procedura che costituisce il cuore del lavoro svolto durante lo stage,
  e conseguentemente di questa tesi, che ha lo scopo di simulare l'evoluzione dello stato delle variabili della rete,
  campionando casualmente una \textit{traiettoria}.

  \begin{algorithm}
    \DontPrintSemicolon
    \SetKwFunction{FMain}{CTBN\_Sample}
    \SetKwProg{Fn}{Procedure}{:}{}
    \Fn{\FMain{$t_{end}$}}{
      t $\leftarrow$ 0, 
      $\sigma$ $\leftarrow$ $\emptyset$\;
      \;
      \ForEach{$\textit{X} \in \mathbf{X}$}
        {Scegli $x(0)$ dato $\bm{\theta}^{\mathcal{B}}_{X \; | \; \mathbf{Pa}_{\mathcal{B}}(X)}$}
      \;
      \While {true}{
        \ForEach{$\textit{X} \in \mathbf{X}$}{
          \uIf{Time(X) \textbf{is} \textit{undefined}}{
            $\Delta t$ $\leftarrow$ $random\_exponential(q_{x(t) \; | \; u_x(t)})$\;
            $Time(X)$ $\leftarrow$ t + $\Delta t$
          }
        }
        \;
        $X_{tr}$ $\leftarrow$ $arg \; min_{Y \in \bm{X}}$[Time(Y)]\;
        t $\leftarrow$ Time($X_{tr}$)\;
        \;
        \eIf{$t \geq t_{end}$}{
          \KwRet $\sigma$\;
        }{
          $x_{tr}(t) \leftarrow random\_multinomial(\bm{\theta}_{x(t) \; | \; u_x(t)})$\;
          Aggiungi $\langle t, x_1(t), x_2(t), ..., x_{tr}(t), ..., x_n(t) \rangle$ a $\sigma$
        }
        \;
        
        \ForEach{$Y \in \mathbf{Cd}(X_{tr}) \cup X_{tr}$}{
          $Time(Y) \leftarrow undefined$\;
        }
      }
    }
    
    \captionof{algocf}{Procedura CTBN\_Sample}
    \label{alg:ctbn_sample}
  \end{algorithm}

  Il metodo \textit{CTBN\_Sample}, quindi, restituisce come risultato una traiettoria che parte dall'istante $t_0 = 0$
  e tale per cui l'istante temporale associato all'ultima occorrenza che è stata campionanta è minore o uguale al parametro $t_{end}$
  specificato.\

  \subsection{Interpretazione dello pseudo-codice}
  Questa sezione si occupa di analizzare e spiegare le istruzioni dell'Algoritmo \ref{alg:ctbn_sample}.\\
  La procedura si occupa, in fase di inizializzazione nelle righe da \textit{2} a \textit{6}, 
  di assegnare il valore $0$ alla variabile $t$, che rappresenta l'istante temporale corrente, e di definire la traiettoria $\sigma$
  come insieme vuoto $\emptyset$.\\
  Si è deciso di rappresentare la traiettoria come un insieme, quindi senza un ordinamento, in quanto
  ciascun elemento appartenente ad esso include l'informazione temporale corrispondente. 
  Una rappresentazione equivalentemente consisterebbe in una sequenza ordinata di n-uple.\\
  A riga \textit{5} si valorizzano le variabili all'istante $t=0$, infatti $\theta^{\mathcal{B}}_{X \; | \; \mathbf{Pa}_{\mathcal{B}}(X)}$
  indica la distribuzione di probabilità iniziale della variabile $X$ dati i suoi genitori.\\ 
  La parte più significativa della procedura, ovvero il campionamento vero e proprio della 
  traiettoria, è data dalla ripetizione del blocco di istruzioni definite nel corpo
  del ciclo \textit{while} che inizia a riga \textit{8}.\\
  Premesso che con $Time(X)$ ci si riferisce all'istante temporale in cui $X$ effettuerà la prossima transizione di stato,
  la prima operazione che viene svolta all'inizio di ciascuna iterazione ha lo scopo di determinare $Time(X)$
  per ogni variabile $X$ tale per cui esso non sia già definito. Come verrà spiegato successivamente,
  l'insieme delle variabili il cui tempo di transizione è indefinito è costituito
  dalla variabile che ha cambiato stato all'iterazione immediatamente precedente a quella corrente, unitamente alle 
  variabili che dipendono da essa (in altre parole le sue figlie).\\
  Per calcolare $Time(X)$ si somma al tempo corrente $t$ un valore $\Delta t$ estratto in 
  modo casuale da una distribuzione esponenziale di parametro $q_{x(t) \; | \; u_x(t)}$.
  La distribuzione esponenziale si addice a questo scopo (e in generale ai domini in cui si rappresenta il tempo rimanente
  prima che si verifichi un evento) in quanto gode della proprietà di \textit{assenza di memoria},
  che stabilisce che il tempo trascorso non influisce sulla probabilità che l'evento si verifichi.
  Tale proprietà viene formalizzata come segue:
  \begin{equation}
    P(X > m + n \; | \; X > m) = P(X > n)  
  \end{equation}
  Questo paramentro consiste chiaramente nel coefficiente della CIM che si trova alla
  riga associata allo stato attuale della variabile, in corrispondenza della diagonale.
  Come spiegato nella trattazione sulle \textit{CIM}, l'interpretazione di questo coefficiente è
  riconducibile alla probabilità di lasciare lo stato corrente.\\
  Acclarato che $Time(X)$ risulta ora definito per ogni nodo appartenente alla rete,
  si procede ad identificare la variabile che verrà coinvolta per prima in un cambio di stato, ovvero quella 
  caratterizzata dal minor tempo di transizione associato. Ciò è formalizzato attraverso 
  l'espressione a riga \textit{15}, denotando con $X_{tr}$ la variabile identificata per la transizione.\\
  Si simula quindi un avanzamento nel tempo, un'evoluzione, aggiornando l'istante temporale corrente
  all'istante in cui avviene la transizione della variabile prescelta. A questo punto,
  se questa progressione implicasse il raggiungimento o eventualmente il superamento del tempo massimo
  $t_{end}$, la procedura terminerebbe ritornando la traiettoria generata fino a quel momento.
  Il raggiungimento del tempo massimo stabilito costituisce quindi la condizione di uscita del loop di generazione 
  della traiettoria.\\
  Se invece questa condizione non è verificata, si procede all'effettivo cambio di stato della
  variabile identificata in precedenza. 
  Nel caso generale di variabili con cardinalità \textit{n}, il nuovo valore viene
  determinato casualmente estraendolo da una multinomiale con parametri $\theta_{x(t) \; | \; u_x(t)}$.
  La distribuzione multinomiale consiste in una generalizzazione della distribuzione binomiale, in cui le variabili
  possono assumere solo i valori 0 e 1, al caso in cui ciascuna variabile può assumere $k$ diversi valori.
  I coefficienti $\theta_{x(t) \; | \; u_x(t)}$ utilizzati per la multinomiale sono interpretabili come l'insieme di probabilità
  che la variabile $X$ effettui una transizione verso ciascuno dei possibili stati, dato l'insieme delle istanze dei genitori $u_x(t)$. 
  Da un punto di vista pratico, questi coefficienti costituiscono la riga della \textit{CIM} corrispondente all'attuale valore della variabile,
  previa normalizzazione (in modo che la somma dei coefficienti che la compongono sia 1).\\
  Poiché la variabile che compie la transizione non può mantenere lo stato corrente, è necessario avere cura di 
  azzerare la probabilità che lo stato estratto corrisponda a quello attuale (ovvero il valore proveniente dal coefficiente che si trova sulla diagonale).\\
  Questo accorgimento, unitamente all'identificazione di una sola variabile per ciascuna iterazione, è necessario per soddisfare 
  il vincolo precedentemente introdotto in relazione al fatto che esattamente una variabile cambi stato in un dato istante.\\
  È possibile asserire che nel caso particolare di variabili binarie, ossia con cardinalità 2, la transizione è 
  implementabile semplicemente negando lo stato corrente tramite la formula

  \begin{equation}
    x_{tr}(t) = 1 - x_{tr}(t-1)  
  \end{equation}

  Una volta realizzata la transizione effettiva di $X_{tr}$, si ottiene una nuova istanza che viene aggiunta alla traiettoria,
  interpretabile come una descrizione dello stato di $\bm{X}$, ovvero di tutte le variabili appartenenti
  alla rete, all'istante temporale corrente \textit{t}. Questa rappresentazione
  viene implementata come una sequenza ordinata in cui si pone in prima posizione
  l'istante temporale a cui si riferisce la situazione descritta, seguito dallo
  stato attuale di tutte le \textit{n} variabili della rete (riga \textit{22}).\\
  Il \textit{foreach} finale a riga \textit{25} si occupa di reimpostare ad \textit{undefined} 
  il tempo di transizione della variabile $X_{tr}$ che ha appena cambiato stato nell'iterazione corrente,
  e di tutte le variabili che hanno un arco entrante che parte da essa, quindi dipendenti.
  Ciò permette di fare in modo che alla successiva iterazione venga determinato un nuovo
  istante temporale per la transizione di queste variabili, come illustrato in precedenza.\\
  L'implementazione effettiva integrata nella libreria \textit{PyCTBN}
  è stata arricchita di alcune estensioni e funzionalità, che verranno descritte
  nella sezione dedicata, la quale si occupa inoltre di mostrare in termini attuativi
  quanto introdotto con lo pseudo-codice.