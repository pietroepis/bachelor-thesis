\chapter*{Introduzione}
  \label{chapter_introduzione}
  \addcontentsline{toc}{chapter}{Introduzione}
  \paragraph{}
  Le Reti Bayesiane a Tempo Continuo (CTBN) sono un modello grafico probabilistico nato per rappresentare in modo efficiente sistemi
  dinamici, in evoluzione, sfruttando i concetti fondanti di altri modelli quali le Reti Bayesiane (BN) 
  e i Processi Markoviani a Tempo Continuo (CTMP).\\
  I numerosi domini il cui stato progredisce al trascorrere del tempo (considerato come una quantità continua) costituiscono la naturale applicazione delle CTBN, 
  che permettono di modellarne l'evoluzione nel tempo.
  Un possibile esempio si colloca nel contesto dei mercati finanziari, in cui la quotazione di uno
  stock varia nel tempo e il cui trend è condizionato da diversi fattori, quali l'andamento di altri titoli correlati
  piuttosto che avvenimenti di carattere politico, sociale o commerciale.\\
  Il percorso di stage è stato quindi dedicato all'implementazione di nuove funzionalità all'interno
  della libreria software \textit{PyCTBN}, sviluppata dal \textit{MAD Lab (Models and Algorithms for Data and Text Mining)}
  allo scopo di mettere a disposizione uno strumento completo ed efficiente per utilizzare
  il framework delle reti Bayesiane a tempo continuo. In particolare, i nuovi moduli sviluppati
  vertono sulla generazione randomica di reti e sul processo di campionamento di traiettorie,
  integrando inoltre utility per l'esportazione su file dei dati prodotti.\\
  Questa tesi prevede un capitolo introduttivo (Capitolo \ref{chapter_concetti_preliminari}), strutturato in modo da fornire le definizioni
  fondamentali ed esporre le notazioni di cui si fa uso nei capitoli seguenti, presentando
  poi i modelli da cui derivano le CTBN. Vengono successivamente illustrate le reti a tempo continuo (Capitolo \ref{chapter_ctbn}), 
  partendo dalla definizione formale fino ad arrivare ad una dettagliata trattazione che spiega dal
  punto di vista teorico i concetti e le procedure su cui si basa il lavoro svolto.
  Infine, l'ultimo capitolo (Capitolo \ref{chapter_pyctbn}) ha lo scopo di documentare la libreria PyCTBN, dedicando ovviamente
  particolare attenzione alle nuove classi sviluppate, e di discuterne i risultati
  in termini di efficienza rispetto a velocità di esecuzione e utilizzo della memoria.