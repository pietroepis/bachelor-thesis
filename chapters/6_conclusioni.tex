
  \chapter*{Conclusioni}
  \label{chapter_conclusioni}
  \addcontentsline{toc}{chapter}{Conclusioni}
  \paragraph{}
  In relazione a quanto osservato è possibile affermare che gli obiettivi iniziali sono
  stati soddisfatti, dando un piccolo contributo allo sviluppo e alla crescita di
  PyCTBN. È stata inoltre mantenuta l'intenzione di rispettare i principi su cui è stata
  sviluppata la libreria, ricorrendo a scelte progettuali che garantissero la flessibilità
  e l'estensibilità del software, astraendosi completamente dal formato dei dataset e
  agevolando ulteriori integrazioni.\\
  Per garantire la qualità del lavoro, tutti i metodi sono stati approfonditamente testati
  con molteplici configurazioni, sperimentando inoltre l'interazione con le componenti già esistenti.
  Allo scopo di favorire la leggibilità, la manutenibilità e innanzitutto l'utilizzo dei
  moduli introdotti, è stata prodotta una documentazione riguardante tutte le classi, i metodi e i 
  relativi parametri, provvedendo inoltre a fornire un esempio pratico di realizzazione dell'intero flusso
  di esecuzione, che prevede la generazione della rete, il campionamento di una traiettoria
  su di essa, l'apprendimento della struttura a partire da tale traiettoria e infine 
  l'esportazione su file dei dati prodotti.\\
  Come già ampiamente sottolineato, la libreria rimane aperta all'introduzione di
  funzionalità aggiuntive che le possano conferire ulteriore completezza, rendendola
  un framework open-source consolidato per lavorare con le reti Bayesiane a tempo continuo.
  Si potrebbe pensare di allineare PyCTBN all'unica alternativa esistente, CTBN-RLE,
  implementando tutti i moduli non ancora presenti e superando le criticità prestazionali peculiari di quest'ultima.
  Alcune funzionalità significative sarebbero ad esempio legate all'inferenza su una rete
  e all'apprendimento con dati non completi.
  Un ulteriore miglioramento possibile, anche se non di banale realizzazione, consisterebbe
  nel prendere in considerazione la proprietà di faithfulness nell'ambito della generazione
  delle reti, che come spiegato nel relativo capitolo è stata attualmente trascurata.\\
  Per quanto riguarda il percoso di stage, è stata un'esperienza altamente formativa,
  che ha convogliato molte delle competenze acquisite nel corso del triennio,
  dando inoltre una dimostrazione dell'applicazione pratica di concetti finora
  ristretti alla sola teoria. È stata anche l'occasione di un primo approccio al 
  mondo della ricerca, cercando di trarre il più possibile dalla competenza, dall'esperienza e dalla
  passione dimostrate da chi mi ha guidato in questo lavoro, fornendomi spunti e
  consigli con grande disponibilità.