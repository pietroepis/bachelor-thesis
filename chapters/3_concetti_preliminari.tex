\chapter{Concetti Preliminari}
  \label{chapter_concetti_preliminari}
  \paragraph{}
  Questo capitolo ha lo scopo di introdurre le nozioni fondamentali su cui si
  basano le Continuous Time Bayesian Network (\textbf{CTBN}).

  \section{Definizioni e Notazione}
  \paragraph{}
  Vengono inizialmente fornite le definizioni essenziali e presentate le notazioni convenzionalmente 
  adottate ed utilizzate nel corso dei capitoli successivi.

  \paragraph{Variabile di Processo} \cite{nodelman-2007}
  Una variabile di processo può essere considerata come una collezione di variabili aleatorie
  indicizzate rispetto al tempo continuo e viene indicata con lettere maiuscole.
  e.g. \textit{E}, \textit{$X_i$}.\\    
  Di conseguenza ci si riferisce alle normali variabili aleatorie con un indice temporale,
  e.g. \textit{$X_{i}(t)$}, \textit{$E_t$}

  \paragraph{Valori delle Variabili}
  Il valore assunto da una variabile è indicato dalla relativa lettera minuscola, e.g. \textit{Z = z}, oppure
  $z_1$, $z_2$, ..., $z_n$ per riferirsi a valori diversi.\\
  L'insieme delle possibili istanze di una variabile \textit{Z} è denotato da \textit{Val(Z)},
  e il numero delle istanze (ovvero la cardinalità dell'insieme) si indica con \textit{Card(Z)}.

  \paragraph{Insiemi di Variabili}
  Insiemi di variabili sono denotati da lettere maiuscole e in grassetto (e.g. \textit{\textbf{E}},
  \bm{$V$}) e i relativi valori con le corrispondenti lettere minuscole (e.g. \textit{\textbf{e}},
  \bm{$v$}).

  \paragraph{Distribuzione di Probabilità}
  La distribuzione di probabilità associata ai valori di una variabile aleatoria \textit{$X(t)$} è denotata
  equivalentemente da \textit{P(X(t))} o da \textit{$P_{x}(t)$}. Nel caso della distribuzione iniziale, ovvero
  quando \textit{t = 0}, si usa scrivere \textit{$P_{x}^{0}$}.\\
  La distribuzione di probabilità condizionata di \textit{X(t)} dato \textit{X(s)} è denotata da
  $P(X(t) \; | \; X(s))$

  \paragraph{Genitori}
  Dato un grafo $\mathcal{G}$, si indica con $\mathbf{Pa}_{\mathcal{G}}(X_i)$ l'insieme dei nodi
  genitori della variabile \textit{$X_i$}.\\
  Similmente, l'insieme dei nodi figli di \textit{$X_i$} è rappresentato da $\mathbf{Cd}_{\mathcal{G}}(X_i)$.

  \paragraph{Densità di un Grafo} \cite{graphs}
  Dato un grafo orientato $\mathcal{G} = (V, \; E)$, dove $V$ sono i nodi ed $E$ gli archi, la densità $d$ è definita come
  il rapporto tra il numero di archi effettivi ($\lvert E \lvert$) e il numero massimo di archi che $\mathcal{G}$ avrebbe 
  se fosse un grafo completo. Formalmente:
  \begin{equation} \label{eq:density}
    d = \frac{\vert E \vert}{\vert V \vert (\vert V \vert - 1)}
  \end{equation}

  \section{Bayesian Networks} 
  Il concetto principale su cui si basano le \textbf{Reti Bayesiane (BN)} è l'indipendenza condizionale, 
  definita come segue.

  \begin{definition}[Indipendenza Condizionale] \cite{conditional-independence}
    Due variabili aleatorie \textit{$B_1$} e \textit{$B_2$} si dicono condizionalmente indipendenti,
    dato un insieme di variabili aleatorie \textit{\textbf{C}}, se e solo se
    \begin{equation}
      P(B_1 \; | \; \bm{C}, B_2) = P(B_1 \; | \; \bm{C})
    \end{equation}
  \end{definition}

  Intuitivamente ciò significa che, per ogni possibile valore delle variabili \textit{\textbf{C}},
  \textit{$B_1$} e \textit{$B_2$} non hanno influenza l'una sull'altra,
  o analogamente che la loro influenza è mediata dalle variabili in \textit{\textbf{C}}.\\
  È ragionevole pensare che la maggior parte delle variabili in una certa rete non abbia un'influenza
  diretta su gran parte delle altre, ma piuttosto che ciascuna variabile sia direttamente condizionata 
  da un sottoinsieme limitato delle altre variabili.\\
  È a questo punto possibile definire formalmente una rete Bayesiana.

  \begin{definition}[Bayesian Network] \cite{bn}
    Una Rete Bayesiana $\mathcal{B}$, definita su un insieme di variabili aleatorie \textit{\textbf{B}},
    è costituita da:
    \begin{itemize}
      \item Un grafo diretto aciclico (DAG) $\mathcal{G}$ in cui ciascun nodo è associato
        ad una variabile \textit{$B_i$} $\in$ \textit{\textbf{B}}
      \item Per ciascuna variabile \textit{$B_i$}, una distribuzione di probabilità condizionata (CPD) 
        $P(B_i \; | \; \mathbf{Pa}_{\mathcal{G}}(B_i))$ che formalizza la dipendenza diretta di \textit{$B_i$}
        dall'insieme dei suoi genitori.
    \end{itemize}
  \end{definition}

  Una rete Bayesiana rappresenta quindi una distribuzione congiunta di probabilità,
  che sfruttando la \textit{Chain Rule} può essere espressa come prodotto di fattori:
  \begin{equation}
    P(B_1, B_2, ..., B_n) = \prod_{i=1}^{n} P(B_i \; | \; \mathbf{Pa}_{\mathcal{G}}(B_i))  
  \end{equation}

  Il grafo $\mathcal{G}$ associato alla rete incorpora anche informazioni riguardanti l'indipendenza
  condizionale tra le variabili, in particolare il fatto che una variabile \textit{$B_i$} $\in$ \textit{\textbf{B}}
  è indipendente dalle variabili che non sono sue discendenti, dato $\mathbf{Pa}_{\mathcal{G}}(B_i)$

  \subsection{Dynamic Bayesian Networks}
  Le \textbf{Reti Bayesiane Dinamiche (DBN)} sono uno strumento che permette di modellare
  processi stocastici a \textbf{tempo discreto} (o \textit{time sliced}).
  Lo stato della rete a ciascun istante di tempo, quindi, è descritto da un insieme
  di variabili aleatorie \textit{\textbf{D}}. 
  Indicizzando gli insiemi di variabili rispetto al tempo, si ottiene che l'intera collezione di variabili è 
  perciò costituita dall'insieme \{$\bm{D_0}$, $\bm{D_1}$, ..., $\bm{D_n}$\},
  dove ciascun elemento $D_i$ denota l'insieme delle variabili della rete all'istante \textit{i}.\\
  Formalmente, una rete Bayesiana dinamica è definita come segue.

  \begin{definition}[Dynamic Bayesian Network]  \cite{nodelman-2007}
    Una rete Bayesiana dinamica è una coppia $\langle\mathcal{B}_0, \mathcal{B}_{2T}\rangle$,
    dove $\mathcal{B}_0$ è la rete Bayesiana basata sulle variabili in $\bm{D_0}$, e $\mathcal{B}_{2T}$
    (chiamato \textbf{2TBN}, 2 Time Slice Bayesian Network) fornisce un modello di transizione dall'istante \textit{i} all'istante \textit{i+1}.\\
    In particolare, $\mathcal{B}_{2T}$ consiste in una rete Bayesiana costituita dalle variabili in $\bm{D_{i+1}}$ 
    condizionate dalle variabili in $\bm{D_i}$.\\
    Gli archi di questa rete vengono classificati in due categorie:
    \begin{itemize}
      \item \textbf{Intra}-time-slice: sono consentiti solo tra variabili appartenenti allo stesso 
        istante temporale, e.g. (X, Y) per X, Y $\in$ \bm{$D_i$}
      \item \textbf{Inter}-time-slice: connettono una variabile nel contesto di un istante temporale ad
        una dell'istante successivo (eventualmente anche la stessa), e.g. (X $\in$ \bm{$D_i$}, X $\in$ \bm{$D_{i+1}$})
    \end{itemize}
  \end{definition}

  \section{Continuous Time Markov Processes}
  Un ulteriore elemento necessario per poter introdurre le reti Bayesiane a tempo continuo consiste nei
  processi Markoviani a tempo continuo (\textbf{Continuous Time Markov Processes}), che derivano dall'aggiunta
  di un'assunzione (assunzione di Markov) ai processi stocastici a tempo continuo.
  
  \begin{definition}[Processo stocastico a tempo continuo] \cite{stochastic-processes}
    Si definisce processo stocastico (o aleatorio) a tempo continuo una collezione di variabili aleatorie
    indicizzate rispetto al tempo $\bm{t \in [0, \infty)}$.\\
    Si consideri la variabile \textit{X} in relazione a tutti i valori di \textit{t} come 
    un'unica variabile (\textbf{Process Variable}).
  \end{definition}

  Definiamo quindi l'assunzione di Markov, da applicare al concetto di processo stocastico a tempo continuo.

  \begin{assumption}[Assunzione di Markov] \cite{ctmp}
    Si assume che, in un processo stocastico, il valore futuro sia indipendente da quello passato, dato il valore presente.\\
    Formalmente, un processo \textit{X} soddisfa l'assunzione di Markov se e solo se
    \begin{equation}
      P(X_{t+\Delta t} \; | \; X_t, X_s) = P(X_{t+\Delta t} \; | \; X_t)
    \end{equation}
    \[
      \forall s, t \; | \; s<t<\infty
    \]
  \end{assumption}

  \paragraph{}
  In altre parole, l'assunzione di Markov implica che la transizione verso un nuovo stato dipenda esclusivamente dallo stato immediatamente precedente,
  e non da come si è giunti ad esso, garantendo quindi la cosiddetta \textit{memoryless property}.
  
  \begin{definition}[Processo Omogeneo] \cite{markov-chains}
    Un processo Markoviano si dice omogeneo se la probabilità di transizione $P(X_{t+1} = b \; | \; X_t = a)$ non dipende
    dall'istante temporale \emph{t} ma esclusivamente dagli stati \emph{a} e \emph{b}.\\
    In un processo di Markov omogeneo, quindi, la probabilità di transizione dallo stato \emph{a} allo 
    stato \emph{b} è costante nel tempo.
  \end{definition}

  \paragraph{}
  È ora possibile fornire la definizione di \textbf{traiettoria}, elemento centrale nel contesto di questa tesi, 
  che verrà ripreso dettagliatamente nel seguito.

  \begin{definition}[Traiettoria]
    Una specifica istanza dei valori di una variabile $X_t$, per ogni \textit{t}, prende il nome di traiettoria.  
  \end{definition}

  \paragraph{Rappresentazione del processo Markoviano}
    Un processo di questo tipo è generalmente definito da una \textit{distribuzione iniziale} e da una
    \textbf{matrice di intensità di transizione}, concetto che sarà fondamentale anche nell'ambito delle
    CTBN.\\
    È previsto che gli elementi della matrice che si trovano sulla diagonale principale abbiano valore negativo, mentre
    gli altri devono essere non negativi.
    Le matrici devono inoltre soddisfare il vincolo tale per cui la somma degli elementi di ciascuna riga sia pari a zero.\\
    Un esempio di matrice è quindi il seguente:

  \[
    Q_X = 
    \begin{bmatrix}
      -q_1 & q_{12} & \dots & q_{1n}\\
      q_{21} & -q_2 & \dots & q_{2n}\\
      \vdots & \vdots & \ddots & \vdots\\
      q_{n1} & q_{n2} & \dots & -q_n
    \end{bmatrix} 
  \]
  \[
    dove \; q_i = \sum_{j \neq i} q_{ij}, \; con \; q_{ij} \geq \; 0 \; \forall \; i, j \in [1, n]
  \]

  \paragraph{}
  L'interpretazione della matrice è che i valori \textit{$q_i$}, ovvero quelli che si trovano sulla
  diagonale, forniscono la probabilità istantanea di lasciare lo stato \textit{$x_i$}, mentre il generico
  elemento in posizione \textit{(i, j)} indica la probabilità istantanea di effettuare una transizione dallo stato \textit{i} allo stato \textit{j}.\\
  Si parla di probabilità istantanea, e non di probabilità in senso assoluto, in quanto 
  i valori che costituiscono una riga non sommano ad uno. È però possibile assimilare un coefficiente al concetto tradizionale di probabilità
  effettuando una normalizzazione rispetto alla riga a cui esso appartiene. Per considerare un
  coefficiente $q_{ij}$ come probabilità $p_{ij}$, quindi, si procede come segue:
  \begin{equation}
    p_{ij} = \frac{q_{ij}}{\sum_{k \in \{1...n\} \setminus i}q_{ik}}
  \end{equation}

  Complessivamente, pertanto, la matrice descrive il comportamento istantaneo del processo \textit{X} rispetto
  allo stato in cui si trova nell'istante corrente.

  \subsection{Conditional Markov Processes}
  \paragraph{}
  Un successivo passo verso la definizione delle CTBN è consentito dal concetto di \textbf{processo Markoviano condizionato} (Conditional Markov Process) \cite{nodelman-2007},
  che è un modello di processo Markoviano \textit{non omogeneo} in cui le intensità di transizione
  variano con il variare del tempo \textit{t}, ma non in funzione di \textit{t}.\\
  Le intensità di transizione di una variabile all'istante \textit{t} dipendono invece dal valore che 
  altre variabili assumono all'istante \textit{t}, con queste ultime che a loro volta evolvono come processi Markoviani.
  
  \subsubsection{Conditional Intensity Matrix}
  \paragraph{}
  Sia \textit{X} una variabile con dominio \textit{Val(X)} = \{$x_1$, $x_2$, ..., $x_n$\} e sia
  $\mathbf{P} = \mathbf{Pa}(X)$.\\
  \textit{X} evolve nel tempo come processo Markoviano le cui variazioni sono condizionate dallo
  stato delle variabili in $\mathbf{P}$, che a loro volta evolvono al variare del tempo.\\
  La necessità di esprimere questa dipendenza introduce il concetto di Conditional Intensity Matrix (\textbf{CIM}),
  sulla base delle matrici di intensità già mostrate. Una CIM, infatti, è sostanzialmente composta da un insieme
  di matrici di intensità, una per ciascuna istanza \textbf{p} delle variabili \textbf{P}.\\
  Nel caso limite in cui \textit{X} non ha genitori, la sua CIM corrisponde esattamente ad una semplice
  matrice di intensità.\\
  Una notazione che evidenzia la variazione dei parametri della matrice in funzione dei valori di 
  \textbf{p} è la seguente (in cui chiaramente valgono gli stessi vincoli sui valori della diagonale): 

  \vspace{6mm}
  \[
    Q_{X|P} = 
    \begin{bmatrix}
      -q_1(P) & q_{12}(P) & \dots & q_{1n}(P)\\
      q_{21}(P) & -q_2(P) & \dots & q_{2n}(P)\\
      \vdots & \vdots & \ddots & \vdots\\
      q_{n1}(P) & q_{n2}(P) & \dots & -q_n(P)
    \end{bmatrix} 
  \]
  \vspace{6mm}

  Una \textit{CIM}  è sempre di forma quadrata, e in particolare di dimensione pari al numero di stati
  che la variabile \textit{X} a cui essa è associata può assumere, ovvero la sua cardinalità \textit{Card(X)}.
